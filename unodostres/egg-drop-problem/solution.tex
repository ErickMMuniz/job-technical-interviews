\documentclass{article}
\usepackage[utf8]{inputenc}
\usepackage[spanish]{babel}
\usepackage{amsmath} % Necesario para el entorno equation

\title{Solución al Problema del Huevo y el Edificio}
\author{Erick Muñiz Morales}
\date{}

\begin{document}

\maketitle

\section*{Estrategia Óptima y Número de Intentos}

El objetivo es encontrar el número mínimo de intentos ($k$) requeridos para determinar la resistencia de un huevo en un edificio de 100 pisos. La estrategia óptima para el peor caso, diferente a la búsqueda binaria, utiliza una progresión aritmética decreciente.

\subsection*{Principio de la Progresión Aritmética}

La estrategia consiste en elegir saltos de piso que se reducen en una unidad con cada éxito. Si se requieren $k$ intentos en total, el primer intento debe realizarse desde el piso $k$. Si el huevo sobrevive, el siguiente salto es de $k-1$ pisos, luego $k-2$, y así sucesivamente, hasta un salto de 1 piso. Esto garantiza que si el huevo se rompe en cualquier intento, los intentos restantes serán suficientes para una búsqueda lineal exhaustiva en el rango anterior.

\subsection*{Cálculo del Mínimo de Intentos ($k$)}

El número total de pisos cubiertos por $k$ intentos es la suma de los primeros $k$ enteros positivos. Esta suma debe ser mayor o igual a 100:

\begin{equation}
    N_{\text{pisos}} = k + (k-1) + (k-2) + \dots + 1 = \sum_{i=1}^{k} i
\end{equation}

Aplicando la fórmula de la suma de Gauss:

\begin{equation}
    \frac{k(k+1)}{2} \ge 100
\end{equation}


\subsubsection*{Verificación para $k=14$}
Si $k=14$, el número máximo de pisos cubiertos es:
\begin{equation*}
    \frac{14(14+1)}{2} = \frac{14 \times 15}{2} = 105
\end{equation*}
Dado que $105 \ge 100$, \textbf{14 intentos} es el número mínimo requerido.

\subsection{Solución óptioma}

En ese caso, la mejor solución que encontramos es una estrategia binaria. Es decir, partiendo de un piso y haciendo una búsqueda en una de las mitades del edificio. 



\end{document}
